% Created 2022-02-28 Mon 10:54
% Intended LaTeX compiler: pdflatex
\documentclass[11pt]{article}
\usepackage[utf8]{inputenc}
\usepackage[T1]{fontenc}
\usepackage{graphicx}
\usepackage{grffile}
\usepackage{longtable}
\usepackage{wrapfig}
\usepackage{rotating}
\usepackage[normalem]{ulem}
\usepackage{amsmath}
\usepackage{textcomp}
\usepackage{amssymb}
\usepackage{capt-of}
\usepackage{hyperref}
\author{Frank Lu}
\date{\today}
\title{Main Config}
\hypersetup{
 pdfauthor={Frank Lu},
 pdftitle={Main Config},
 pdfkeywords={},
 pdfsubject={},
 pdfcreator={Emacs 27.2 (Org mode 9.5)}, 
 pdflang={English}}
\begin{document}

\maketitle
\tableofcontents


\section{Quantum Processes}
\label{sec:org8d55c96}
\section{Review of the Process Tensor}
\label{sec:org24b325a}
\section{Restricted Boltzmann Machine}
\label{sec:orgaf84c34}
\section{Theoretical Results}
\label{sec:org09b813a}
\newcommand{\nH}{N_H}
\newcommand{\NH}{N_H}
\newcommand{\NV}{N_H}

\newcommand{\as}{a_1a_2 \dots a_{\nh}}
\newcommand{\ones}{1, 1, \dots, 1}
\newcommand{\ks}{k_1k_2 \dots b_{\nh}}

\newsavebox{\mybox}
\newenvironment{Notes}
{\begin{lrbox}{\mybox}\begin{minipage}{\textwidth}}
{\end{minipage}\end{lrbox}\fbox{\usebox{\mybox}}\\}

Hi!

\subsection{Arguing that RBMs are intrinsically acausal}
\label{sec:org6f39cbc}

Here is some mathematical definition.

\(\frac{3}{4}\)
\section{Machine Learning Approach}
\label{sec:org49857af}
\end{document}
