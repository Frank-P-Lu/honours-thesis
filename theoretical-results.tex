% Created 2022-02-28 Mon 11:02
% Intended LaTeX compiler: pdflatex
\documentclass[11pt]{article}
\usepackage[utf8]{inputenc}
\usepackage[T1]{fontenc}
\usepackage{graphicx}
\usepackage{grffile}
\usepackage{longtable}
\usepackage{wrapfig}
\usepackage{rotating}
\usepackage[normalem]{ulem}
\usepackage{amsmath}
\usepackage{textcomp}
\usepackage{amssymb}
\usepackage{capt-of}
\usepackage{hyperref}
\author{Frank Lu}
\date{\today}
\title{Config}
\hypersetup{
 pdfauthor={Frank Lu},
 pdftitle={Config},
 pdfkeywords={},
 pdfsubject={},
 pdfcreator={Emacs 27.2 (Org mode 9.5)}, 
 pdflang={English}}
\begin{document}

\maketitle
\tableofcontents

% Theoretical results section
\newcommand{\nH}{n_H}
\newcommand{\nV}{n_V}
\newcommand{\NH}{N_H}
\newcommand{\NV}{N_V}

\newcommand{\as}{a_1a_2 \dots a_{\nh}}
\newcommand{\ones}{1, 1, \dots, 1}
\newcommand{\ks}{k_1k_2 \dots b_{\nh}}

\newsavebox{\mybox}
\newenvironment{Notes}
{\begin{lrbox}{\mybox}\begin{minipage}{\textwidth}}
{\end{minipage}\end{lrbox}\fbox{\usebox{\mybox}}\\}

Hi!

\section{Arguing that RBMs are intrinsically acausal}
\label{sec:orgec442dd}
\subsection{Solving for elements of the contracted tensor}
\label{sec:orgaffba52}

\begin{Notes}
In fact, we can solve for specific elements of the \(D\) tensor.

\begin{align}
    &= \sum_{s \in S} \prod_{h=1}^{\nh} C^{h,\nv}_{1, s} \label{correction} \\
    &= \sum_{s \in S} \prod_{h=1}^{\nh} e^{ (s -1) \frac{V_{\nv}}{\nv} } \\
    &= \sum_{s \in S} e^{ (s -1) \frac{\nh V_{\nv}}{\nv} }
\end{align}

\end{Notes}
\end{document}
